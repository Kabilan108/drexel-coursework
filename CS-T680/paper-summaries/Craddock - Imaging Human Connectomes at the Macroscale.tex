\documentclass[11pt, letterpaper]{article}

% Packages
\usepackage[utf8]{inputenc}
\usepackage[letterpaper, margin=1in]{geometry}
\usepackage{fancyhdr}
\usepackage{lastpage}

% Header and footer styles
\pagestyle{fancy}
\lhead{CS T680}
\chead{Paper Summary}
\rhead{Tony Kabilan Okeke}
\setlength{\headheight}{14pt}


% Document
\begin{document}

\begin{center}
    \large\textbf{Imaging Human Connectomes at the Macroscale.\\
                  Craddock, R., Jbabdi, S., Yan, CG. et al.}
\end{center}

The term connectome refers to a multiscale construct that describes different 
regions of the brain, their anatomical connections and functional interactions. 
The connectome can be analyzed at microscopic, macroscopic and mesoscopic 
resolutions, with macroscopic resolutions being the most useful the mapping and 
annotation of human connectomes with cognitive and behavioral associations. 
Magnetic Resonance Imaging (MRI) serves as the dominant modality used for 
macroconnectomics due to its relative safety and spatial resolution. In the 
study of structural connectivity, Diffusion-weighted MRIs (dMRI) provide 
cubic-milimeter resolution depictions of white-matter tracts which allow for 
the identification of axon trajectories; functional MRIs (fMRI) allow for the 
investigation of the brain's functional architechure. The paper described 
connectomes mathematically as graphs of interactions between regions of the 
brain. An initial challenge with the creation of these "brain-graphs" is the 
identification of nodes; there is no consensus on the optimal approach for 
defining brain regions, atlases tend to be selected based on the regions of 
interest in the study being conducted; each cortical region in the selected 
atlas serves as node the connectome (brain graph).

The primary tool in the investigation of structural connectivity has been dMRI 
data. In dMRI, a series of images are acquired by measuring the deflection of 
water molecules due to changes in the direction of the surrounding magnetic 
field; several images are generated at different unique directions. The 
combination of these images allows for the estimation of the directions of 
axonal fibers in each voxel; this information is then combined to form a 
tractogram - a collection of all the identified axonal tracts which can be 
viewed as a 3D rendering of white matter tracts. These trajectories 
(streamlines) are linked to corresponding brain regions (nodes) using a variety 
of tractography approaches. These techniques allow for the inference of the 
existence (or absence) of connections between nodes, as well as the 
determination of the strengths of those connections - these serve as the edges 
(and weights) of the connectome (brain graph). Tractography is not without it's 
limitations however, noise in the dMRI data can affect the accuracy of the 
measurements. The models used to determine streamline direction are also unable 
to resolve certain axon arrangements accurately.

Functional connectivity tends to be less intuitive than its counterpart and is 
also more challenging to estimate. BOLD-based (Blood Oxygenation Level Dependent) 
fMRIs are the most widely used technique for inferring functional connectivity. 
BOLD fMRIs are measured using ultrafast imaging sequences that detect relatice 
concentrations of deoxyheamoglobin (due to its paramagnetic nature); this 
provides a measure of neuronal activation. The fMRI images undergo several 
preprocessing steps in order to construct the functional connectome.

Once the connectomes (brain graphs) have been constructed, the next goal is to 
develop annotations for relevant higher-order cognitive processes, 
neuropsychiatric diagnoses or other phenotypic variables. These annotations are 
typically inferred through categorical and dimensional analyses comparing the 
connectomes of several individuals across time or different treatments. The 
connectomes can be compared using a variety of graph theory approaches, the 
simplest of which involves treating each graph as a collection of edges and 
performing statistical analyses one edge at a time; this approach does not take 
into account the interactions between edges. Alternatively, multivariate 
classification and regression techniques can be applied to evaluate the 
relationship between the connectome as a whole and associated phenotypic 
variables with a single statistical test. Predictive modelling techniques can 
also be used in the identification of distinct connectivity patterns that are 
predictive of specific pheotypic variables.

The study of connectomics offers several potential benefits including the 
expansion of our knowledge of brain architecture and its links to behavior and 
cognition, as well as advancements in clinical tools for the diagnosis and 
treatment of different neurological disorders. 

In my opinion, the paper provides an excellent introduction to the field of 
connectomics. The descriptions of dMRI and fMRI, and the associated data 
preprocessing steps necessary for each were very insightful. My main critique of 
the paper is that the authors did not go into great depth in discussing the 
analytical tools used in connectomics.

\end{document}