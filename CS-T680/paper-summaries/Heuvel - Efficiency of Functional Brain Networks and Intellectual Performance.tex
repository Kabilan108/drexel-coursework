\documentclass[11pt, letterpaper]{article}

% Packages
\usepackage[utf8]{inputenc}
\usepackage[letterpaper, margin=1in]{geometry}
\usepackage{fancyhdr}
\usepackage{lastpage}

% Header and footer styles
\pagestyle{fancy}
\lhead{CS T680}
\chead{Paper Summary}
\rhead{Tony Kabilan Okeke}
\setlength{\headheight}{14pt}

    
% Document
\begin{document}

\begin{center}
    \large\textbf{Efficiency of Functional Brain Networks and Intellectual Performance\\
                  Martijn P. van den Heuvel et. al.}
\end{center}

\begin{itemize}
    \setlength\itemsep{0.05em}
    
    \item {Information is constantly processed and integrated between \textit{functionally linked},
           \textit{spatially distributed} brain regions}
    \item {
        Functional connections in the brain network are organized in an efficient \textit{small-world} manner; 
        this suggests:
        \begin{itemize}
            \setlength\itemsep{0.05em}
            \item high-level of local neighborhood clustering which is responsible for efficient local information processing
            \item existence of several long-distance connections that increases global communication efficiency (across sub-networks (regions)>
        \end{itemize}
    }
    \item {The organization of the brain network raises the question of whether there is a relationship between
           the efficiency of functional connections and intelligence}
    
\end{itemize}


\noindent\textbf{Materials and Methods}
\begin{itemize}
    \setlength\itemsep{0.05em}
    \item Subjects - Healthy (No psychiatric history); 14 Male; 5 Female; Age$=29\pm7.8$
    \item {Intelligence Scores - Dutch WAIS-III test - gives a standardized full-scale IQ based on
           several subtests that assess verbal IQ and nonverbal (performance) IQ}
    \item {Functional Connectomes - constructed using correlations between spontaneous brain signals of
           different brain regions at rest from fMRI time series}
    \item {Preprocessing - fMRI time series were realigned to correct for small head movements; coregistered 
           T1 images for anatomical overlap; normalized to standard-space; bandpass filtered (0.01 - 0.1Hz)}
    \item {
        Graph analysis
        \begin{itemize}
            \setlength\itemsep{0.05em}
            \item Connectoms were constructed from all cortical and subcortical brain voxels (~9500 nodes) with
                  connections between all functionally linked voxels
            \item $|Functional~Connectivity|_{ij} =$ zero-lag correlation between voxelwise resting-state BOLD time series;
            \item {Voxels $i$ and $j$ are \textit{functionally linked} when the zero-lag correlation > threshold, 
                   $T$ ($0.05 \leq T \leq 5$)}
            \item Computed key graph theory measures including \textbf{clustering coefficient, $C$} and 
            \textbf{characteristic path length, $L$}
            \item {Computed the ratios: $\gamma = C/C^{random}$ and $\lambda = L/L^{random}$.
                   Where $C^{random}$ and $L^{random}$ are properties of a random organized network of similar size.}
            \item {Small-world organization is characterized by $\gamma >> 1$ and $\lambda \approx 1$}
        \end{itemize}
    }
    \item {Intelligence -  $\gamma$ and $\lambda$ values were correlated with the full-scale IQ scores of the
           participants}
    \item {Intelligence - The total number of connections $k$ was also correlated with IQ scores to determine 
           whether intelligence was correlated with the number of connections in a brain network.}
    \item {
        To identify which brain regions had the strongest association between network organization and IQ,
        the normalized path lengths of each node were correlated with IQ separately.
        \begin{itemize}
            \item {Individual normalized path length reflects how closely (efficiently) connected a node is to 
                   other nodes in the network}
            \item {A correlation-coefficient map was constructed showing which voxels (nodes) exhibited a
                   significant (p < 0.05) association between IQ and normalized path length}
        \end{itemize} 
    }
\end{itemize}


\noindent\textbf{Results}
\begin{itemize}
    \setlength\itemsep{0.05em}
    \item The functional connectomes exhibited small-world organization for $0.3\leq T \leq0.5$
    \item {For higher $T$, a significant negative correlation was found between normalized characteristic path 
           length $\lambda$ and IQ}
    \item {IQ scores showed no correlation with the number of total number of connections $k$; this indicates that the
           observed correlation between IQ and $lambda$ was not related to potential variations in overall connectivity}
    \item No significant correlation was found between the clustering coefficient $\gamma$ and IQ
    \item {The most prominent correlation between individual normalized path length (for each node/voxel) and IQ was
           in the medial prefrontal gyrus, posterior cingulate gyrus, and bilateral inferior parietal regions}.
\end{itemize}


\noindent\textbf{Discussion}
\begin{itemize}
    \setlength\itemsep{0.05em}
    \item {The study found a strong association between global communication efficiency and intellectual performance.\\
           (Normalized characteristic path length $\lambda$ was strongly negatively correlated with IQ) \\
           This suggests that more efficiently functionally connected brains show a higher level of intellectual
           performance}
    \item {Normalized clustering coefficient $\gamma$ and number of connections $k$ were found to not be
           significantly correlated with IQ \\
           This suggests that intelligence is not directly related to the level of local information processing
           or to the total number of functional connections}
    \item {The strong correlation between $\lambda$ and $IQ$ suggests that intelligence is more related to the
           efficiency of the organization of global connections and the efficiency of information integration
           between different brain regions}
    \item {Strong associations between ID and individual normalized path length were found in several brain regions.
           This is important since efficient hub nodes are more likely to have a strong effect on global network 
           efficiency than less connected nodes.}
    \item {\textit{post-hoc} analyses were performed to examine the association between $\lambda$ and IQ. \\
           The association between $\lambda$ and the subscales of the WAIS test was examined. $\lambda$ was most 
           related to PIQ and perceptual organization index (POI)}
\end{itemize}


\end{document}