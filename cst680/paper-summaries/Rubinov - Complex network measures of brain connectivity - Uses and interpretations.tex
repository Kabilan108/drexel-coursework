\documentclass[11pt, letterpaper]{article}

% Packages
\usepackage[utf8]{inputenc}
\usepackage[letterpaper, margin=1in]{geometry}
\usepackage{fancyhdr}
\usepackage{lastpage}

% Header and footer styles
\pagestyle{fancy}
\lhead{CS T680}
\chead{Paper Summary}
\rhead{Tony Kabilan Okeke}
\setlength{\headheight}{14pt}


% Document
\begin{document}

\begin{center}
    \large\textbf{Complex network measures of brain connectivity: Uses and interpretations\\
                  Mikail Rubinov, Olaf Sporns}
\end{center}

Complex network analysis is a branch of mathematics dealing with the study of large, complex real-life networks; 
it provides a reliable way to quantify brain networks using a small set of biologically meaningful and easily 
computable features. Network analysis is also a useful tool in the exploration of relationships between functional 
and structural connectomes. In this paper, the authors review some of the most common network measures used to 
quantify brain networks. They discuss different network measures and their significance in the brain network. 
For example, \textit{degree centrality} is a measure of the "importance" of a node in a network, while 
\textit{betweenness centrality} is a good measure of the ability of a node to influence the flow of information 
in a network.\\


\noindent\textbf{Construction of brain networks}

\indent Nodes in large-scale brain networks usually represent brain regions, while links represent anatomical, 
functional, or effective connections. Anatomical connections typically correspond to white matter tracts between 
pairs of brain regions. Functional connections correspond to magnitudes of temporal correlations in activity and 
may occur between pairs of anatomically unconnected regions. Effective connections represent direct or indirect 
causal influences of one region on another and may be estimated from observed perturbations.

Nodes are identified using a combination of brain mapping methods, anatomical parcellation schemes, and measures 
of connectivity. There are different types of links that may be used to connect nodes in a network. Links may be 
binary (only denoting presence/absence of connections) or weighted (including information about relative connection 
strengths). Weak or non-significant links may represent spurious connections and tend to be excluded by if they fall 
below some arbitrary threshold. Links can also be distinguished by their directionality; connectomes constructed 
from Diffusion MRIs contain directionless links while some effective and functional connectomes may contain directed 
links. The nature of nodes and links largely determines the biological interpretation of the network topology so great 
care must be taken in their identification.\\


\noindent\textbf{Measures of brain networks}

Individual network measures may characterize one or several aspects of global and local connectivity. Measures of 
individual elements (nodes or links) reflect the way these elements are embedded in the network, while measures of 
the distribution of those elements provide a global description of the network. Most network measures are influenced 
by the size and topology of the network. The significance of these measures could be established by comparison with 
null-hypothesis networks.

Measures of segregation can be used to quantify the presence of functional clusters within the network; they have 
straightforward interpretations in anatomical and functional networks. Simple measures of segregation are based on 
the number of triangles in the network - more triangles implies more segregation. More sophisticated measures of 
segregation describe the size and composition of interconnected regions.

Measures of functional integration characterize the brain's ability to combine specialized information from 
distributed brain regions and are based on the concept of \textit{paths} - sequences of distinct nodes and links 
which represent potential routes of information flow between brain regions. The most common measure of functional 
integration is the \textit{characteristic path length} of the network (the average shortest path length between 
all pairs of nodes).

Measures of segregation and integration do not capture the local patterns of connectivity that are particularly 
diverse in directed networks. These patterns (motifs) can be anatomical or functional. The significance of a motif 
is determined by its frequency of occurrence. Motif fingerprinting (determining the frequencies of different motifs 
around an individual node) can be used to infer the functional role of a brain region. Network motifs are generally 
not used in the analysis of undirected networks due to the lack of local undirected connectivity patterns.

Measures of node centrality assess the importance of individual nodes on a variety of criteria. A common measure of 
centrality is \textit{degree} and it has a simple biological interpretation:  "nodes with a high degree are interacting 
structurally or functionally with may other nodes in the network". Measures of centrality such as 
\textit{Closeness Centrality} and \textit{Betweenness Centrality} are based on the notion that central nodes participate 
in several short paths within a network and thus, act as important controls of information flow.\\


\noindent\textbf{Network comparison}

The authors also discuss the difficulty associated with comparing different types of networks. One common issue is that 
functional networks are often denser than anatomical networks, due to the fact that they can contain connections between 
anatomically unconnected regions. This makes global comparisons between the two types of networks more difficult. Other 
factors that may affect comparisons of network topology include degree and weight distributions. The authors also 
note that the relationship between structural and functional brain connectivity is nontrivial, and this makes 
comparisons of functional networks between different subjects more difficult.

\end{document}